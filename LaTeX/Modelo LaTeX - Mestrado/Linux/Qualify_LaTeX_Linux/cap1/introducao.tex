\chapter[INTRODUÇÃO]{\textbf {Introdução}}
\section{CONTEXTUALIZAÇÃO E FORMULAÇÃO DO PROBLEMA}

A área médica tem recebido muita contribuição da ciência da computação, seja pela
inclusão de computadores nos hospitais e centros de saúde, seja pela aplicação de técnicas
computacionais na organização das informações médicas. O uso da TI na área médica
adicionou novos desafios para a integração dos computadores, equipamentos médicos e
sistemas de informação existentes nos hospitais. Atualmente já se fala em um ambiente
hospitalar digital e integrado \cite{huang}.

Levando-se esse conceito para a prática radiológica tem-se a radiologia digital e a
mamografia digital. A transição da radiologia analógica para a digital implica na utilização
da TI para compartilhar e distribuir informações médicas dentro e fora do ambiente
hospitalar \cite{huang}.

A radiologia digital se utiliza de mecanismos eletrônicos de gerenciamento das
informações radiológicas, em oposição aos mecanismos analógicos, baseados em papel e
filme radiográfico \cite{dreyer}. Em um ambiente de radiologia digital, os
registros de pacientes, o agendamento de procedimentos, as imagens e laudos médicos são
produzidos e utilizados digitalmente.

A transição da radiologia convencional para a digital alcançou a prática da
mamografia, recebendo a denominação de mamografia digital que se caracteriza pela
aquisição, processamento e armazenamento eletrônico das imagens da mama \cite{baert}.

O uso da mamografia digital tem sido amplamente difundido, pois se acredita que essa
prática representa um avanço na detecção precoce do câncer de mama \cite{trambert},
além de reduzir a quantidade de radiação ionizante a que o paciente é submetido \cite{huang}.
Outro fator de interesse é a capacidade e facilidade em se comparar
exames de múltiplas modalidades, por exemplo, de mamografia digital e ultra-sonografia
digital, o que contribui para um diagnóstico mais preciso do câncer de mama. Além disso,
a capacidade de armazenamento e recuperação facilitada das imagens permitem ao
radiologista utilizar softwares para o auxílio ao diagnóstico (CADx - Computer-Aided
Diagnosis) e para a detecção de câncer de mama (CADe - Computer-Aided Detection), o
12que pode resultar em uma melhor interpretação dos achados mamográficos \cite{baert}.

No contexto da mamografia digital, a integração tecnológica impõe novos desafios
enfrentados pelos administradores dos hospitais, radiologistas, técnicos e fabricantes de
equipamentos. Esses desafios podem ser resumidos no esforço para se planejar e implantar
uma comunicação sincronizada e eficiente entre o sistema de informações radiológicas
(RIS), os mamógrafos digitais e o servidor de imagens médicas (PACS). A combinação
correta desses elementos representa a transição da prática da mamografia convencional,
baseada em filmes, para a mamografia digital completa ou FFDM \cite{trambert}.

Os benefícios de uma solução FFDM são vários. Do ponto de vista dos
administradores, FFDM significa a diminuição de custos se comparada à utilização de
soluções isoladas de mamografia. Sob o ponto de vista dos profissionais de saúde, FFDM
significa o acesso rápido às informações dos pacientes em qualquer lugar e a qualquer
hora. Finalmente, sob o ponto de vista do paciente, a integração pode significar um estudo
clínico mais detalhado e, conseqüentemente, um diagnóstico mais preciso porque pode
contar com o trabalho conjunto de diversos especialistas espalhados na unidade hospitalar \cite{trambert}.

Por outro lado, o uso de mamógrafos desprovidos de uma integração completa pode
gerar duplicidade de informações e, conseqüentemente, o aumento do re-trabalho e a
diminuição da segurança na execução dos procedimentos médicos. Isso porque ocorrem
intervenções manuais freqüentes e não padronizadas no RIS e nos aparelhos de
mamografia. Essas intervenções manuais têm por objetivo, registrar as informações sobre
os pacientes e sobre os exames realizados e, por isso, são imprescindíveis na execução dos
procedimentos radiológicos. Entretanto, quando as informações médicas não são
gerenciadas de maneira correta, problemas de ineficiência e segurança dos dados ocorrem
com freqüência \cite{grimes}. Um exemplo disso seria a inclusão de uma identificação
de paciente inválida quando da execução do procedimento de mamografia. Esse erro
dificultaria a localização do exame realizado, o que poderia causar a repetição do
procedimento e uma nova exposição do paciente à radiação ionizante do aparelho.

No caso da prática de mamografia a integração das tecnologias é de vital importância,
pois representa a capacidade de se executar um rastreamento mais apropriado dos achados
13mamográficos desde o primeiro até o último exame realizado. Assim, torna-se
imprescindível o armazenamento e recuperação adequados das imagens e, também, das
informações sobre a condição médica do paciente.

Envolvido na necessidade de integração entre os equipamentos de mamografia digital,
o RIS e o PACS, encontra-se o engenheiro clínico, um profissional que atua no meio
hospitalar com o objetivo de melhorar o atendimento ao paciente através da aplicação de
seus conhecimentos de engenharia e de gestão \cite{acce1}.

As mudanças ocasionadas pela inclusão da TI no meio médico têm alterado a prática
da engenharia clínica que deixou de ser uma simples atividade associada à manutenção e
reparo de equipamentos, passando a ser uma área que lida com questões mais estratégicas,
relacionadas também ao uso de padrões para a garantia da interoperabilidade das
tecnologias médicas e para a gestão efetiva das informações de saúde \cite{acce2}.

Como resultado dessa mudança de abordagem, observa-se que o engenheiro clínico e
os profissionais de TI que atuam nos hospitais têm trabalhado em conjunto com o objetivo
de atingir níveis de eficiência mais elevados \cite{acce2}.

Assim, os engenheiros clínicos e os profissionais de TI combinam suas melhores
habilidades com o objetivo de reduzir a ocorrência de erros médicos e, conseqüentemente,
melhorar a segurança do paciente.
\cite{grimes} em The challenge of integrating the
healthcare enterprise, aponta como um dos desafios da indústria médica e, mais
especificamente, dos engenheiros clínicos, “a informatização da saúde e a interconexão dos
dispositivos e sistemas médicos”. No que diz respeito à informatização, ele declara que “a
informatização dos dispositivos médicos é uma conseqüência natural das necessidades de
rapidamente adquirir, processar e apresentar uma quantidade crescente e variada de
informações médicas”. Em relação à interconectividade, Grimes (2005) afirma que “a
interconexão entre diferentes aparelhos médicos pode levar a uma troca mais direta,
precisa e rápida de informações sobre saúde”.

Diante do exposto acima, este trabalho apresenta um estudo, realizado em um
ambiente que executa o procedimento de mamografia digital, com o objetivo de propor um
mecanismo adequado para a integração entre RIS, PACS e mamógrafos digitais.