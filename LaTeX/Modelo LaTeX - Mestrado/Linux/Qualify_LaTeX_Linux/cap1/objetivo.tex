\section[OBJETIVOS]{OBJETIVOS}

\subsection[Objetivo Geral]{\textbf{Objetivo Geral}}
Este trabalho tem por objetivo principal propor um processo de trabalho para a
implantação adequada de um ambiente FFDM em hospitais e clínicas de mamografia.
\subsection[Objetivo Específicos]{\textbf{Objetivos Específicos}}
Os objetivos deste trabalho podem ser detalhados segundos dois aspectos ou áreas de
interesse: engenharia clínica e tecnologia da informação.

Quanto à engenharia clínica, este trabalho se propõe a:
\begin{itemize}
 \item Estudar o funcionamento e características dos equipamentos de mamografia digital;
 \item Estudar e analisar os padrões e normas nacionais e internacionais relacionadas à aplicação de FFDM em clínicas e hospitais;
 \item Propor um processo de trabalho que auxilie os engenheiros clínicos na implantação de soluções integradas de mamografia digital, RIS e PACS.
\end{itemize}

Quanto à tecnologia da informação, este trabalho se propõe a:
\begin{itemize}
 \item Estudar e analisar soluções de arquitetura de software e sua aplicação no contexto
da FFDM, mais especificamente, de implantação de PACS em clínicas e hospitais
que praticam a mamografia digital;
\item Estudar e analisar o padrão DICOM e sua aplicação no contexto da FFDM;
\item Estudar e analisar a estrutura IHE e sua aplicação no contexto da FFDM;
\item Propor modelos para o mapeamento adequado de informações trocadas entre o RIS,
PACS e os mamógrafos digitais.

\end{itemize}
