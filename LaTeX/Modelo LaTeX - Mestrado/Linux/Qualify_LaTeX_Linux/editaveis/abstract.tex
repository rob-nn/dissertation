\begin{resumo}[Abstract]
 \begin{otherlanguage*}{english}

 \begin{center}
\textbf{\imprimirTitle}
\end{center}

\begin{flushleft}
\footnotesize
\textbf{Author: \imprimirautor}\\
\textbf{Supervisor: Prof(a). \imprimirorientador} \\
\textbf{Co-supervisor: \imprimirmembroCoorientador} \\
\textbf{Post-Graduation Program in Biomedical Engineering – Qualify of Master Degree}
\textbf{Brasília, Month of Year.}\newline

EXAMPLE:
\end{flushleft}
 
The transition from the Mammography based on films to the Digital Mammography or
FFDM (Full-field Digital Mammography) is not a simple task, involving solely the
replacement of a few electronic components and the addition of new computers and
software to the radiology workspace. The work routine of physicians, radiology
technicians and the supporting staff are all of them affected, because the work procedures
must be restructured in order to comply with the communication protocols and data
integration rules used in the field of medicine nowadays. Moreover, the new technologies
added to the radiology workspace, for instance, the RIS (Radiology Information Systems)
and PACS (Picture Archiving and Communications Systems, imply the obligation of
elaborating a more efficient strategy for adequately sharing the medical information
between the information systems of radiology and the digital mammography equipment.
Consequently, this essay offers a study involving the integration between RIS, PACS and
digital mammography equipments aiming at helping clinical engineers and IT
professionals in the planning, preparation and implementation of an FFDM environment
in conformity with the data communication standards used in the medicine field nowadays
and, more specifically, in the digital mammography practice.


   \vspace{\onelineskip}
 
   \noindent 
   \textbf{Key-words}: \imprimirpalavrachaveumingles, \imprimirpalavrachavedoisingles, 
			\imprimirpalavrachavetresingles, \imprimirpalavrachavequatroingles.
 \end{otherlanguage*}
\end{resumo}
