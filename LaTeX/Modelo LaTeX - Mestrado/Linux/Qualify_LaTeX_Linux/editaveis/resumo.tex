\begin{resumo}

\begin{center}
\textbf{\imprimirtitulo}
\end{center}

\begin{flushleft}
\footnotesize
\textbf{Autor: \imprimirautor }\\
\textbf{Orientador(a): Prof(a). \imprimirorientador }\\
\textbf{Co-orientador(a): \imprimirmembroCoorientador} \\
\textbf{Programa de Pós-Graduação em Engenharia Biomédica – Qualificação de Mestrado }\\
\textbf{\imprimirlocal \imprimirdata }



EXEMPLO:
\end{flushleft}

A transição da mamografia baseada em filmes para a mamografia digital completa ou
FFDM (Full-field Digital Mammography) não é uma tarefa simples que envolve
unicamente a substituição de alguns componentes eletrônicos e a adição de computadores e
softwares no ambiente de radiologia. O trabalho dos médicos, técnicos de radiologia e
profissionais de apoio são afetados, pois o fluxo de trabalho deve ser reestruturado de
modo a atender aos protocolos de comunicação e integração de dados utilizados na área
médica. Além disso, as novas tecnologias adicionadas ao ambiente radiológico, por
exemplo, RIS (Radiology Information Systems) e PACS (Picture Archiving and
Communications Systems), imputam a necessidade de elaborar uma estratégia mais
eficiente para o compartilhamento adequado das informações médica entre os sistemas de
informação radiológica e os equipamentos de mamografia digital. Diante disso, este
trabalho apresenta um estudo envolvendo a integração entre RIS, PACS e equipamentos de
mamografia digital com o objetivo de auxiliar os engenheiros clínicos e profissionais de TI
(Tecnologia da Informação) no planejamento, preparação e implantação de um ambiente
FFDM em conformidade com os padrões de comunicação de dados utilizados na área
médica e, mais especificamente, na prática da mamografia digital.

\vspace{\onelineskip}
    
 \noindent
 \textbf{Palavras-chaves}: \imprimirpalavrachaveum, \imprimirpalavrachavedois, 
			    \imprimirpalavrachavetres, \imprimirpalavrachavequatro.
\end{resumo}

