\chapter[FUNDAMENTAÇÃO TEÓRICA]{\textbf {fundamentação teórica}}
\section[MAMOGRAFIA DIGITAL]{MAMOGRAFIA DIGITAL}
\subsection[Mamografia e o câncer de mama]{\textbf{Mamografia e o câncer de mama}}

Segundo o INCA (Instituto Nacional de Câncer), “o número de casos novos de câncer de
mama esperados para o Brasil em $2010$ será de $49.240$, com um risco estimado de $49$ casos
a cada $100$ mil mulheres”. Essa estatística confirma que o câncer de mama é o segundo
tipo de câncer mais freqüente no mundo e o mais comum entre as mulheres. A cada ano,
cerca de $22\%$ dos novos casos de câncer em mulheres são de mama e por isso, os governos
do mundo inteiro têm investido em políticas para a detecção precoce da doença \cite{inca}. $\ldots$

Entretanto, no Brasil o relatório de estimativa de câncer de 2009 aponta que: “Apesar
de ser considerado um câncer de relativamente bom prognóstico, se diagnosticado e tratado
oportunamente, as taxas de mortalidade por câncer de mama continuam elevadas no Brasil,
muito provavelmente porque a doença ainda é diagnosticada em estádios avançados”.

A equação da continuidade na forma diferencial, presente na Eq. (\ref{cont}), representa o
princípio de conservação da massa em um escoamento e um exemplo de equação em LaTeX.
Onde $\rho$ é a densidade, $t$ é o tempo e $\vec{u}$ a velocidade do fluido.
\begin{equation}
\label{cont}
\frac{\partial \rho}{\partial t} + \nabla \ldotp (\rho \vec{u}) 
\end{equation}