\chapter[TRABALHOS FUTUROS]{\textbf{TRABALHOS FUTUROS}}
Como trabalhos futuros sugerem-se os seguintes: (i) a validação do processo proposto por
este estudo; (ii) a melhoria do processo proposto para inclusão das modalidades de ultra-
sonografia e ressonância magnética; (iii) a construção de um software para integração das
tecnologias RIS, PACS e mamógrafos digitais; (iv) a elaboração de uma estrutura
padronizada para armazenamento e recuperação de laudos de mamografia no PACS; (v) a
elaboração de uma estrutura padronizada para armazenamento e recuperação de
informações BI-RADS do PACS; (vi) o estudo para extração de informações do RIS e do
PACS para efeitos processamento estatísticos diversos envolvendo a evolução do câncer
em uma população de indivíduos.

Quanto à validação do processo proposto por este estudo, recomenda-se que o mesmo
seja executado pelo menos duas unidades de saúde, uma pública e outra privada, de modo
a se identificar a sua aderência em realidades tão distintas. Isso leva também à necessidade
de construção de um software de integração envolvendo as tecnologias RIS, PACS e
mamógrafos digitais o que propiciaria uma comparação deste produto com algumas
soluções fornecidas no mercado (Figura 37). Além disso, a construção de um software
como este representaria uma validação adicional deste estudo, uma vez que não haveria a
dependência de soluções de mercado para a validação do processo de trabalho sugerido
neste documento.