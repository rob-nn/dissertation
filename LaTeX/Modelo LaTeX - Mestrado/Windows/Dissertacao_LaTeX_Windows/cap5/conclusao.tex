\chapter[DISCUSSÃO E CONCLUSÃO]{\textbf{DISCUSSÃO E CONCLUSÃO}}
Através deste estudo foram realizadas investigações envolvendo o uso de tecnologias e
padrões para a implantação de um ambiente FFDM integrado. Essas investigações geraram
como resultado, a proposta de um processo de trabalho para a condução de projetos de
integração de RIS, PACS e mamógrafos digitais. Da idealização dessa proposta, por sua
vez, considera-se os seguintes pontos de discussão: (i) a participação de engenheiros
clínicos e profissionais de TI nos projetos de integração; (ii) a complexidade dos projetos
de integração; (iii) a dificuldade de aderência das tecnologias aos padrões DICOM e IHE
(declaração de conformidade); (iv) a disseminação do padrão DICOM e IHE no Brasil; (v)
a necessidade de evolução do DICOM e do IHE; e (vi) os cuidados na execução de
projetos de integração entre RIS, PACS e equipamentos de mamografia.


\cite{grimes} em Clinical Engineering: the challenge of change questiona se os
engenheiros clínicos de hoje estão preparados para as mudanças tecnológicas que estão
ocorrendo na área médica, especialmente aquelas associadas à necessidade de integração,
comunicação e distribuição de informações sobre saúde, dentro e fora das unidades
hospitalares. O estudo realizado pôde comprovar que o esforço de integração é grande,
conforme afirmado por Grimes (2005), e deve ser distribuído entre todos os funcionários
que participam do procedimento médico que se deseja aperfeiçoar, sejam eles, médicos,
profissionais do setor administrativo, gestores, engenheiros clínicos e profissionais de TI.

