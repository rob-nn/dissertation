\chapter[METODOLOGIA]{\textbf {metodologia}}

\section[AMBIENTE DE ESTUDO]{AMBIENTE DE ESTUDO}
Este estudo foi realizado em uma clínica de radiologia da cidade de Brasília que oferece
serviços de mamografia digital. Para a realização dos procedimentos de mamografia, a
clínica conta com o apoio de médicos e técnicos radiologistas, além de profissionais que
atuam no setor administrativo, para a recepção de pacientes e entrega de laudos médicos.
Atualmente existem 110.523 pacientes registrados e a média de atendimento para exames
de mamografia digital é de 1095 pacientes por mês ou 36 pacientes diários, o que indica
uma elevada utilização do serviço.

A clínica avaliada neste estudo é uma das poucas unidades de saúde em Brasília que
possui toda a estrutura tecnológica de um ambiente FFDM convencional, ou seja: RIS,
PACS e mamógrafo digital. Entretanto, embora estas tecnologias estejam em pleno
funcionamento, não se encontram integradas, o que propiciou a realização deste estudo.

Além disso, a clínica tem um diferencial, em relação a outros centros de mamografia
existentes em Brasília, que é o de funcionar como uma unidade de saúde que tem interesse
na realização de pesquisas científicas envolvendo o rastreamento, diagnóstico e evolução
do câncer de mama. Isso propiciou uma melhor análise dos dados com o objetivo de
preservar a maior quantidade de informações médicas possíveis no RIS e no PACS.

O RIS em funcionamento na clínica é um produto comercial, fornecido por uma
empresa brasileira. Ele provê recursos de agendamento de consultas e exames, recepção de
pacientes, emissão de laudos médicos, controle de faturamento, estoque e compras, além
da extração de informações gerenciais envolvendo os procedimentos de saúde executados
na clínica.

O PACS em uso na clínica foi fornecido por uma empresa americana e compreende o
servidor PACS e uma estação stand-alone para visualização de imagens. O servidor PACS
possui $1$ $TB$ (terabyte) de capacidade de armazenamento e atualmente cerca de $60\%$ dessa
capacidade está em uso. Adicionalmente, a clínica conta com um storage de $4$ $TB$ de
capacidade que é utilizado para guarda de backups dos dados do PACS e do RIS.

Além do RIS e PACS, a clínica analisada possui um mamógrafo digital, utilizado tanto
para o rastreamento quanto para o diagnóstico de câncer de mama, o que significa dizer
que é possível a realização das incidências padrão (craniocaudal e médio-lateral-oblíqua),
além das incidências adicionais com compressão, angulação e aproximação (magnification) 
diferenciadas. O mamógrafo digital em uso acompanha também uma estação de aquisição e uma estação de diagnóstico.

\section[DELIMITAÇÃO DO ESTUDO]{DELIMITAÇÃO DO ESTUDO}
Este estudo considerou apenas a execução de projetos de integração de equipamentos de
mamografia digital, RIS e PACS. Assim, atividades relacionadas à avaliação e compra
dessas tecnologias não foram analisadas, embora acredita-se que os resultados obtidos no
estudo podem auxiliar aqueles que desejarem adquirir os produtos mencionados.

Além disso, a metodologia de trabalho adotada não procurou analisar as atividades
associadas à gestão e manutenção das tecnologias estudadas, nem tão pouco, descrever
todo o processo de implantação dessas tecnologias.