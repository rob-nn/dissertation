% Configurações de aparência do PDF final
% alterando o aspecto da cor azul
\definecolor{black}{RGB}{0,0,0}

% informações do PDF
\makeatletter
\hypersetup{
     	%pagebackref=true,
		pdftitle={\@title}, 
		pdfauthor={\@author},
    	pdfsubject={\imprimirpreambulo},
	    pdfcreator={LaTeX with abnTeX2},
		pdfkeywords={abnt}{latex}{abntex}{abntex2}{trabalho acadêmico}, 
		colorlinks=true,       		% false: boxed links; true: colored links
    	linkcolor=black,          		% color of internal links
    	citecolor=black,        		% color of links to bibliography
    	filecolor=magenta,      		% color of file links
		urlcolor=black,
		bookmarksdepth=4
}
\makeatother
% O tamanho do parágrafo é dado por:
\setlength{\parindent}{1.25cm} % Controle do espaçamento entre um parágrafo e outro.

\setlength{\parskip}{0.2cm}  % tente também \onelineskip

% Remova o comentário abaixo para espaçamento simples
%\SingleSpacing

%\setlength{\mathindent}{0.6cm} % Altera o recuo das equações de tiver o pacote fleqn em documentclass

\counterwithout{equation}{chapter} %Faz com que as equações sejam numeradas 1,2,3,...

\counterwithout{table}{chapter} %Faz com que as equações sejam numeradas 1,2,3,...


\counterwithout{footnote}{chapter} %Faz com que as notas de rodapé sejam numeradas 1,2,3,...

%\setlength\afterchapskip{\baselineskip} % Espaçamento entre o título e o texto

\renewcommand{\ABNTEXchapterfontsize}{\fontsize{14}{14}\selectfont} %Letra e tamnaho da fonte do sumário
\renewcommand{\ABNTEXsectionfontsize}{\fontsize{14}{14}\selectfont}
\renewcommand{\ABNTEXsubsectionfontsize}{\fontsize{14}{14}\selectfont}
