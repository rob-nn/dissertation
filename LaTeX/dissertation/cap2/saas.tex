\section{SOFTWARE COMO SERVIÇO} 

A idéia de software como serviço é bastante difundida atualmente. 
Redes sociais, serviços de busca, serviços de \emph{streaming} de vídeo, são amplamente usados por todos. 
\cite{Fox2012} define software como serviço, como o software que entrega software e dados como serviços sobre a \emph{Internet}, usualmente via um programa como um \emph{browser} que roda num dispositivo cliente local, em detrimento de código binário que precisa ser instalado e que roda totalmente no dispositivo.

\cite{Fox2012} cita várias vantagens tanto para usuários quanto para desenvolvedores de software. São elas:

\begin{enumerate}
	\item Desde de que os usuários não necessitam instalar a aplicação, eles não precisam se preocupar se possuem um hardware específico, ou se possuem uma versão específica de sistema operacional;
	\item Como os dados associados ao serviço geralmente são mantidos com o serviço, os usuários não precisam em fazer \emph{bakcups}, ou os perderem devido a mal funcionamento, ou até mesmo os perderem devido ao extravios;
	\item Quando um grupo de usuários querem coletivamente interagir sobre os mesmos dados, software como serviço é um veículo natural;
	\item Faz mais sentido manter grandes quantidades de dados centralizados e mater o acesso remoto a estes.
	\item Os desenvolvedores podem controlar as versões de software e sistema operacional que executam o serviço. Isto evita problemas de compatibilidade com distribuição do software devido ao grande número de plataformas que os usuários possuem. Além disso, é possível que o desenvolvedor teste novas versões do software usando uma pequena fração dos usuários temporariamente, sem causar distúrbios na maioria dos usuários;
	\item Como os desenvolvedores controlam a versão de execução do software, eles podem mudar até mesmo a plataforma dos mesmos, desde que não violem as API's de interface com o usuário.
\end{enumerate}

	Para quem duvida do poder do software como serviço, é só observar que produtos consagrado como o \emph{Microsoft Office} já possuem versão como serviço, no caso o \emph{Microsoft Office 365}. Outros exemplos seriam o \emph{Twitter}, \emph{Facebook}, entre outros.



