\subsection{LICENÇA MIT}
\label{mit_sec}

Por meio de software livre, grandes projetos de software tornaram-se realidade. Existem muitas licenças de software livre, cada uma com suas vantagens e disvantagens. Como este trabalho trata da construção de um software, foi decidido que ele seria livre, e que inclusive poderia ser usado para fins comerciais. O tipo de licença escolhida foi a MIT, que é uma das mais permissivas. O texto na integra é mostrado a seguir \cite{MIT2015}:

\begin{citacao}


\emph{The MIT License (MIT)}

\emph{Copyright (c) year copyright holders}

\emph{Permission is hereby granted, free of charge, to any person obtaining a copy}
\emph{of this software and associated documentation files (the "Software"), to deal}
\emph{in the Software without restriction, including without limitation the rights}
\emph{to use, copy, modify, merge, publish, distribute, sublicense, and/or sell}
\emph{copies of the Software, and to permit persons to whom the Software is}
\emph{furnished to do so, subject to the following conditions:}

\emph{The above copyright notice and this permission notice shall be included in}
\emph{all copies or substantial portions of the Software.}

\emph{THE SOFTWARE IS PROVIDED "AS IS", WITHOUT WARRANTY OF ANY KIND, EXPRESS OR}
\emph{IMPLIED, INCLUDING BUT NOT LIMITED TO THE WARRANTIES OF MERCHANTABILITY,}
\emph{FITNESS FOR A PARTICULAR PURPOSE AND NONINFRINGEMENT. IN NO EVENT SHALL THE}
\emph{AUTHORS OR COPYRIGHT HOLDERS BE LIABLE FOR ANY CLAIM, DAMAGES OR OTHER}
\emph{LIABILITY, WHETHER IN AN ACTION OF CONTRACT, TORT OR OTHERWISE, ARISING FROM,}
\emph{OUT OF OR IN CONNECTION WITH THE SOFTWARE OR THE USE OR OTHER DEALINGS IN}
\emph{THE SOFTWARE.}

\end{citacao}
