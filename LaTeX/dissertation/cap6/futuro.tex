\chapter[TRABALHOS FUTUROS]{\textbf{TRABALHOS FUTUROS}}
Como este foi um projeto com um intuito de plantar uma semente, não faltarão trabalhos futuros para complementá-lo, aumentá-lo ou mesmo expândi-lo para outras áreas além da análise de marchar.
\begin{enumerate}
	\item O trabalho mais urgente a ser feito é pegar a versão atual, testá-la o máximo possível, corrigir os bugs encontrados e disponibilizá-la na web;
	\item Criar um módulo de usuários e instituições;
	\item Criar um filtro para retirar ruídos da animação;
	\item Inserir o padrão ouro nos gráficos para efeito de comparação;
	\item Incluir um controle do tipo \emph{slide} para controlar a animação;
	\item Criar o componente de animação como uma diretiva do \emph{angular-js}, para reutilização em outros pontos do projeto;
	\item Criar no módulo de simulação uma ferramenta para modelagem de sinais de entrada, métodos de processamento e sinais de saída, baseados nos dados da base de documentos;
	\item Permitir adicionar e visualizar foto do paciente;
	\item Permitir adicionar e visualizar vídeos das amostras de marcha coletadas;
	\item Habilitar o protocolo \emph{HTTP Auth} nas requisições feitas a \emph{web API};
	\item Habilitar protocolo \emph{HTTPS} entre servidor \emph{web} e \emph{browser} cliente;
	\item Utilizar um componente do tipo \emph{data picker} nos campos de datas;
	\item Implementar um detector automático de ciclo de marchar, assim não será necessário o usuário informar o início e fim do ciclo;
	\item Permitir cadastrar protocolos de coleta por câmeras e fazer a detecção automática dos mesmos, assim o usuário não necessitará nomear marcadores;
	\item Permitir coletar dados de plataforma de força e criar gráficos;
	\item Permitir coletar dados de \emph{IMUs} e criar gráficos;
	\item Permitir coletar dados de \emph{EMGs} e criar gráficos;
	\item Permitir coletar dados de eletrogoniômetros;
	\item Criar suporte a várias língua, começando com português e inglês;
	\item Implementar outros algoritmos de sistemas inteligentes, como \emph{PCA}, \emph{Kmeans}, \emph{SVM}, \emph{MLP}, entre outros, integrando estes algoritmos a ferramenta de modelagem de sinais, permitindo se fazer classificações e regressões. Por exemplo, usando-se \emph{SVM} é possível fazer a detecção de quedas, ou usando o Kmeans é possível detectar os momento distintos no uso de uma plataforma de força. Aqui o que vai imperar é a criatividade do pesquisador, que terá nas mãos uma ferramenta visual para fazer estas simulações. A evolução desse módulo acontecerá quando a simulações forem úteis o suficiente para auxiliar nos diagnósticos de patologias na marcha. Com a tecnologia atual de aprendizado de máquina, as possibiliades são muito atrativas.
	\item Integrar o módulo de simulação com plataformas de infraestrutura com serviço, usando para isso ferramentas como Hadoop ou Spark;
	\item Tornar a função de execução de simulações assíncronas;
	\item Criar um módulo gestor da execução das simulações, com opções de parar, pausar, continuar, alocar mais recursos, enfileiramento.
\end{enumerate}

