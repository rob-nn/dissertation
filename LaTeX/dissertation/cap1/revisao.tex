\section[REVISÃO DA LITERATURA]{revisão da literatura}
Foram usados os seguinte serviços \emph{web} para o levantamento bibliográfico deste trabalho:
\begin{itemize}
	\item \emph{IEEE Xplore Digital Library} (\href{http://ieeexplore.ieee.org}{http://ieeexplore.ieee.org});
	\item \emph{PubMed} (\href{http://www.ncbi.nlm.nih.gov/pubmed}{http://www.ncbi.nlm.nih.gov/pubmed});
	\item Portal de Periódicos CAPES/MEC (\href{http://periodicos.capes.gov.br}{http://periodicos.capes.gov.br}).
\end{itemize}

Foram utilizadas as seguintes chaves de pesquisa em cada um dos serviços acima: ``\emph{Gait Analysis Software}".

Os artigos considerados mais relevantes para o trabalho foram escolhidos, levando-se em consideração, entre outros tópicos, a descrição de características interessantes a serem implementadas no software a ser desenvolvido.

Quando o assunto se trata de análise de marcha, a obra mais aclamada, inclusive citada em muitas das referências pesquisadas, é \citeonline{Perry2010}.
Como sugerido por \citeonline{Malas}, esta é uma obra obrigatória a qualquer um que deseje estudar análise de marcha.

Em \citeonline{Vieira2015} um sistema de análise e classificação de marcha é proposto como alternativa a soluções de mercado mais caras.
A proposta inicial é coletar dados a partir de marcadores posicionados no corpo do paciente, através de câmeras de vídeo, classificando padrões de marcha com aprendizado de máquina.

Em \citeonline{Duhamel2004} é apresentada uma ferramenta para melhorar a confiabilidade de curvas para um paciente, classificar pacientes em determinadas populações e comparar populações.
Trata-se de uma ferramenta estatística para análise de marcha.

Detecções de eventos do ciclo de marcha, são características interessantes para um software de análise de marcha. 
Em \citeonline{Ghoussayni2004} são documentados métodos para detecção de 4 eventos: contato do calcanhar, elevação do calcanhar, contato do dedão do pé e elevação do dedão do pé.

Uma comparação entre dois pacotes distintos para análise de marchar foi realizada por \citeonline{Moraes2003}. 
Neste trabalho dados captados por câmeras e plataformas de força são coletados e passados aos pacotes de software \emph{Kin Trak} e \emph{Ortho Trak}.

Uma amostra de como um software pode ser utilizado para gerar bases de dados de análise de marcha, é visto em \citeonline{Moreno2009}. Neste artigo os autores capturam dados de crianças sadias, afim de obterem padrões para serem utilizados em sistemas de análise de movimentos.

Um sistema de aquisição e análise de marcha, foi desenvolvido e demostrado em \citeonline{Ferreira2009}. 
Neste trabalho, o hardware para captura de dados e o software para análise dos dados, foram desenvolvidos num único projeto.
Com os resultados gerados pelas análises feitas por este projeto, foi possível construir um robô bípede, que apresentou resultados satisfatórios caminhando num ciclo de marcha confortável.

A partir da análise de marcha, é possível criar métodos para se estabelecer o grau de desvio do ciclo de marcha que um paciente pode apresentar.
Em \citeonline{Beynon2010} é apresentado o método \emph{Gait Profile Score}.
O método em si é um bom candidato a funcionalidade em um software de análise de marcha, pois serviria de auxílio clínico ao profissional da área de saúde.
Uma outra funcionalidade inspirada no campo clínico é mostrado em \citeonline{Cippitelli2015}. Neste trabalho os autores 
propõem a automatização do método \emph{Get Up and Go Test}(GUGT), que é usualmente utilizado em análise de marcha no campo da reabilitação.


 
