\section[REVISÃO DA LITERATURA]{revisão da literatura}
Foram usados os seguinte serviços web para o levantamento bibliográfico deste trabalho:
\begin{itemize}
	\item Google Scholar (\href{https://scholar.google.com/}{https://scholar.google.com});
	\item IEEE Xplore Digital Library (\href{http://ieeexplore.ieee.org}{http://ieeexplore.ieee.org});
	\item PubMed (\href{http://www.ncbi.nlm.nih.gov/pubmed}{http://www.ncbi.nlm.nih.gov/pubmed});
	\item Portal de Periódicos CAPES/MEC (\href{http://periodicos.capes.gov.br}{http://periodicos.capes.gov.br}).
\end{itemize}

Foram utilizadas as seguintes chaves de pesquisa em cada um dos serviços acima: "/emph{Gait Analysis Software}" e "/emph{Gait Analysis Open Source}". 

Em \cite{Duhamel2004} é apresentada uma ferramenta para melhorar a confiabiliade de curvas para um paciente, classificação de sujeitos em determinadas populações e comparação entre populações. Trata-se de uma ferramenta estatística para análise de marchar.

