\chapter[INTRODUÇÃO]{\textbf {Introdução}}
\section{CONTEXTUALIZAÇÃO E FORMULAÇÃO DO PROBLEMA}


\begin{comment}
No campo da reabilitação humana, há o subcampo que trata da construção de próteses. 
Estas têm como objetivo substituir a função de algum membro do corpo humano. 
Em especial existem próteses que substituem a função de parte das pernas, abaixo ou acima do joelho, também chamadas de próteses transtibiais para aquelas e transfemurais para estas.

As próteses também podem ser classificadas como passivas ou ativas. As passivas são compostas inerentemente por componentes mecânicos como molas e alavancas. 
Já as próteses ativas possuem mecanismos eletroeletrônicas, que têm como principal função injetar energia no sistema para compensar o gasto metabólico extra, exigido por uma prótese passiva. 
Para as próteses ativas, faz-se necessária a criação de mecanismos de controle para as mesmas.
Este mecanismo pode ser criado, usando-se engenharia de controle tradicional e/ou adaptativa, ou sistemas inteligentes, que permitem visualizar o projeto de controle para próteses \cite{Borjian2008}. 

As técnicas de engenharia de controle \cite{Golnaraghi2010} exigem que se criem modelos cinéticos e/ou cinemáticos para se resolver o problema em questão. 
Em \citeonline{Borjian2008} há um exemplo de tal enfoque. 
Geralmente estes modelos são criados analiticamente, usando-se leis bem compreendidas da física \cite{Garcia2009a}.
Este não é um trabalho fácil, podendo demandar muito tempo para construção e às vezes devido a grande complexidade do sistema, exige um grau maior de simplificação.
Isso pode se tornar um problema sério, até mesmo inviabilizando a solução.

Sistemas inteligentes \cite{Russel2010} abrangem vários tipos de tecnologias.
Dentre elas, sistemas \emph{fuzzy}, sistemas especialistas, agentes lógicos, redes neurais artificiais, etc.
Esta classe de sistemas, pode facilitar bastante a gestão da complexidade, pois dependendo do caso, os modelos são bastante simples de serem construídos, pois não exigiriam toda a complexidade física e matemática da engenharia de controle.

Sistemas \emph{fuzzy} \cite{Lilly2010}, por exemplo, usam como componentes de modelagem variáveis linguísticas.
Exemplificando, estas em um sistema para controle da velocidade de um ventilador poderiam ser resumidas em rápida, média e lenta.

Sistemas baseados em aprendizado de máquina (\emph{machine learning}) \cite{Bishop2006a}, assim como sistemas inteligentes, permitem a criação de sistemas mais simples.
Isto é possível porque estes sistemas não são programados para resolverem um problema, mas sim para “aprenderem” a resolver um problema. 
Por exemplo, é possível desenvolver um sistema baseado em aprendizado de máquina, que aprenderá a partir da coleta de dígitos escritos por várias pessoas, a reconhecer qualquer dígito escrito num papel por quaisquer outras pessoas.

As Redes Neurais Artificiais (RNA) \cite{Haykin2009,Bishop2006a,Russel2010} são classificadas como sistemas baseados em aprendizado de máquina e também como sistemas inteligentes. 
A princípio são sistemas inspirados pela biologia do sistema nervoso central.

O uso de RNAs para controlar próteses transfemurais pode ser um atalho no processo de projeto das mesmas, pois se usando algoritmos de RNAs que já tenham sido especificados e implementados, basta alterá-los para receberem os sinais desejados e produzir as respostas esperadas. 

Porém, o ciclo de marcha possui muitas variações.
Por exemplo, o ciclo confortável, acelerado, subindo aclives, subindo escadas, descendo, etc. Aparentemente é possível criar RNAs para todas estas situações, mas mesmo assim, este é um trabalho hercúleo e que provavelmente exigiria muito poder computacional, o que não seria adequado para uma prótese transfemural ativa, que exige um sistema embarcado eficiente energeticamente.
Com isso em mente, e inspirando-se no trabalho de \citeonline{Sabourin2012}, nota-se que seria possível modelar sistemas \emph{fuzzy} para alterar a saída do sistema conforme a necessidade do ciclo de marchar.
A vantagem de tal configuração é que um sistema \emph{fuzzy} geralmente vai exigir menos cálculos que várias RNAs ou uma RNA com muitos milhares de ativações.

Existem vários tipos de RNA, sendo a mais popular na literatura a \emph{Multi Layer Perceptron} (MLP). 
Esta é um tipo de RNA que tem sua aplicabilidade e eficiência já comprovadas em várias aplicações.
A MLP, porém, tem uma desvantagem para sistemas de grande complexidade.
Ela exige poder computacional considerável, pois necessita fazer muitos cálculos até definir sua saída.
Uma promessa em relação a este tipo de RNA, e que inclusive já possui aplicações no mundo real, é a \emph{Cerebellar Model Articulation Controller} (CMAC) \cite{Albus1975a}.
Esta é uma RNA baseada no cerebelo dos mamíferos.
Sua vantagem em relação à MLP é que na sua forma final, ela pode ser resumida a acessos a tabelas e cálculos triviais e simples para gerar sua saída.

Neste sentido, este trabalho demonstra simulações desenvolvidas a partir de uma RNA CMAC, mostrando que com esta tecnologia é possível controlar uma prótese transfemural ativa durante o ciclo de marcha confortável.
Também é objeto do trabalho, a modificação da saída da RNA CMAC, para que o sistema suporte outros ciclos de marcha através de controladores \emph{fuzzy}.
\end{comment}
