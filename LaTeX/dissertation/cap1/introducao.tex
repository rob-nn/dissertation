\chapter[INTRODUÇÃO]{\textbf {Introdução}}
\section{CONTEXTUALIZAÇÃO E FORMULAÇÃO DO PROBLEMA}
Quando o autor deparou-se com um problema na área de análise de marcha, 
ele seguiu toda uma metodologia para resolver este problema. 
Ao ler artigos desta área, notou que muitas das atividades que ele desempenhou, provavelmente foram desempenhadas por aqueles autores também. 
Na época o autor estava desenvolvendo um software simulador de um joelho, que poderia ser implantado num sistema embarcado para controle de uma possível prótese transfemural ativa.
A metodologia usada naquele trabalho pode ser resumida assim:
\begin{enumerate}
	\item Coletar dados de movimentos usando marcadores passivos fixados no corpo de um paciente. Os dados eram coletados por uma série de câmeras;
	\item Usar o software QTM para converter os dados para o formato \emph{MATLAB}, assim era possível manipulá-los;
	\item Criar um programa para extrair os dados em formatos mais amigáveis à sua manipulação;
	\item Fazer cálculos de ângulos, velocidades angulares, posicionamentos dos marcadores, velocidades dos marcadores;
	\item Fazer gráficos de toda essa informação. Lembrando tudo isso usando \emph{MATLAB};
	\item Persistir toda essa informação para uso futuro;
	\item Criar um algoritmo complexo, que usa certos dados como entrada e algum outro como saída.
	\item Executar várias simulações alterando vários parâmetros até achar uma combinação de parâmetros que simule o sinal de uma forma desejada;
\end{enumerate}

Várias das etapas acima, poderiam ser desenvolvidas num software com interface gráfica, e facilitando e muito o trabalho do pesquisador. Por exemplo, o novo software receberia o arquivo do \emph{QTM} e já criaria, todos os dados de movimentos citados automaticamente, os persistiria numa base de dados e permitiria imprimir gráficos de todos eles.

Este foi o contexto inicial, que impulsionou o desenvolvimento deste trabalho,
mas além deste problema notou-se um potencial a mais, o \emph{QTM} que é o software 
de captura de dados, é muito bom mas é de uso genérico, para utilizá-lo como 
software de análise de marcha, há um trabalho grande a ser feito, o indício disto é 
que a maioria dos pesquisadores com que o pesquisador teve contato no laboratório, 
usavam o mesmo processo, coletavam com o \emph{QTM} e processavam com o \emph{MATLAB}. 
Geralmente a outra opção é usar softwares de análise de marcha específicos como 
software \emph{Kin Trak} e \emph{Ortho Trak}. Além disso todos esses softwares, inclusive o
\emph{QTM} são softwares desktop, que possuem licenças caríssimas o que limita o
uso dos dados pelos pesquisadores, que tem que ir ao laboratório ou ter um computador
com uma licença válida. Daí surgiu outra oportunidade criar o novo software 
como um serviço na web, que pode ir evoluindo ao longo do tempo, ou seja,
recebendo adições de funcionalidades constantemente, até que seja bom o suficiente
para ser usado por qualquer profissional de saúde no globo.

E não é só isso, com o poder de processamento dos dispositivos móveis de hoje, a nova
aplicação também pode resolver problemas de mobilidade, pois pode disponibilizar dados de pacientes
onde e quando o profissional de saúde quiser.


