\begin{resumo}

\begin{center}
\textbf{\imprimirtitulo}
\end{center}

\begin{flushleft}
\footnotesize
\textbf{Autor: \imprimirautor }\\
\textbf{Orientadora: Profa. \imprimirorientador }\\
\textbf{Coorientadora: \imprimirmembroCoorientador} \\
\textbf{Programa de Pós-Graduação em Engenharia Biomédica} \\
\textbf{\imprimirlocal \imprimirdata }
\end{flushleft}

O presente trabalho tem como objetivo implementar um software como serviço, para análise e simulação de marcha humana, baseado num modelo arquitetural em camadas. A grande vantagem de tal software é sua disponibilidade via web e até mesmo em dispositivos móveis. Além disto uma base de dados com dados de marcha humana, será gerada. O sistema ainda conta com um módulo de simulação, que tirará proveito desta base beneficiando pesquisadores no mundo inteiro. Este projeto em questão, é só o começo de algo muito maior, mas análise de movimento baseada em dados espaciais, recuperadas de um software de \emph{motion capture}, já é possível. Também já é possível fazer simulações usando-se  a rede neural artificial CMAC.


\vspace{\onelineskip}
    
 \noindent
 \textbf{Palavras-chaves}: \imprimirpalavrachaveum, \imprimirpalavrachavedois, 
			    \imprimirpalavrachavetres, \imprimirpalavrachavequatro.
\end{resumo}

