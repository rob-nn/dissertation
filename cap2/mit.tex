\subsection{Software Livre}
\label{mit_sec}

Por meio de software livre, grandes projetos de software tornaram-se realidade. 
Existem muitas licenças de software livre, cada uma com suas vantagens e desvantagens. 
Como este trabalho trata da construção de um software, foi decidido que ele seria livre, e que inclusive poderia ser usado para fins comerciais. 
O tipo de licença escolhida foi a do \emph{Massachusetts Institute of Technology (MIT)}, que é uma das mais permissivas. 
O texto na integra é descrito por \citeonline{MIT2015} e foi reproduzido no Anexo \ref{licenca_mit}.

A estratégia para escolha de software livre é que o projeto conta com parcos recursos e um software como este custaria muito para ser desenvolvido. 
Sendo livre é mais provável que outros desenvolvedores contribuam com o software, ajudando com continuidade do projeto, que nas atuais circunstâncias depende muito disto.
Outro bom motivo para o software ser livre é que comunidades inteiras são construídas neste tipo de projeto, o que vem a contribuir mais ainda com a pesquisa e o desenvolvimento da área de análise de marcha.

