\section[PRÓTESES TRANSFEMURAIS ATIVAS]{PRÓTESES TRANSFEMURAIS ATIVAS}

\subsection[Tipos de próteses]{\textbf{Tipos de próteses}}

As próteses constituem uma importante contribuição da engenharia na área da reabilitação humana. Existem próteses para substituição da função de membros exteriores como braços e pernas, até a substituição de secções de esôfagos ou intestinos.

As próteses para substituir as funções das pernas, em especial, têm evoluído bastante ao longo do tempo.
Existem próteses ativas e passivas para este fim. Próteses passivas são constituídas intrinsecamente por elementos mecânicos que não possuem qualquer tipo de suporte energético externo.
Já próteses ativas, são constituídas por elementos mecânicos e eletrônicos, sendo que partes mecânicas da prótese recebem auxílio de energia externa para atuar na função que está sendo substituída, por exemplo, a flexão de um joelho.
A principal vantagem das próteses ativas, sobre as passivas, está na compensação do débito energético que ocorre nas passivas. Esta compensação garante um ciclo de marcha mais confortável, suave e natural ao usuário da prótese ativa \cite{Borjian2008}.

\subsection[Projeto de controle para próteses transfemurais ativas]{\textbf{Projeto de controle para próteses transfemurais ativas}}

Em \citeonline{Sup2008}, demonstra-se o projeto de uma prótese transfemural ativas.
Este trabalho escolheu técnicas de engenharia de controle para controle da prótese.
Basicamente foram criados modelos matemáticos baseados na física clássica para, então, derivar uma equação de transferência que servirá de base para as leis de controle que atuarão na prótese. 

Em \citeonline{Vallery2011} é apresentado um enfoque novo para controle de próteses transfemurais ativas.
Este modelo é baseado em regressões estatísticas, que aplicam os sinais de entrada em um modelo estatístico para então produzir um sinal de saída.
Este é um modelo que pode ser considerado adaptativo.

O terceiro modelo encontrado de projeto de controle para próteses transfemurais ativas, foi o de \citeonline{Borjian2008}. Ele utilizou um controlador \emph{fuzzy}, especificamente com o modelo Takagi-Sygeno-Kang (TSK) \emph{fuzzy}. Este é um modelo adaptativo.



